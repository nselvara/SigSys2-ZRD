%Autor: Simon Walker
%Version: 1.0
%Datum: 23.07.2020
%Lizenz: CC BY-NC-SA

\documentclass[10pt, a4paper]{article}
\title{\fontsize{60}{20} \textbf{SigSys2 ZRD}}
\newcommand{\versioninfo}{v1.0}
\author{S.Walker}


%%%%%%%%%%%%%%%%%%%%%%%%%%%%%%%%%%%%%%%%%%%%%%%%%%%%%%%%%%%%
%Kopf und Fusszeile und Weiter Designeinstellungen
%%%%%%%%%%%%%%%%%%%%%%%%%%%%%%%%%%%%%%%%%%%%%%%%%%%%%%%%%%%%
\usepackage[left=20mm,right=20mm,top=20mm,bottom=15mm,includeheadfoot]{geometry}

\usepackage{fancyhdr}	%Fancyhdr ist ein eingeführtes, eigenständiges Paket zur einfacheren Manipulation von Kopf- und Fußzeile. Es definiert Befehle für den Seitenstil fancy
\usepackage{lastpage}	%Ref­er­ence the num­ber of pages in your LaTeX doc­u­ment through the in­tro­duc­tion of a new la­bel which can be ref­er­enced like \pageref{LastPage} to give a ref­er­ence to the last page of a doc­u­ment. It is par­tic­u­larly use­ful in the page footer that says: Page N of M. 	
\renewcommand{\headrulewidth}{0.25pt} 
\renewcommand{\footrulewidth}{0.25pt}

\fancyhf{}
\fancyhead[L]{SigSys2 ZRD  \tiny{\textit{\versioninfo}}}
\fancyhead[R]{Simon Walker}

\fancyfoot[L]{\today}
\fancyfoot[R]{Seite \thepage { }von \pageref{LastPage}}

\setlength{\parindent}{0pt} %Einrücken von Absätzen verhindern


%%%%%%%%%%%%%%%%%%%%%%%%%%%%%%%%%%%%%%%%%%%%%%%%%%%%%%%%%%%%
% Verwendete Packete
%%%%%%%%%%%%%%%%%%%%%%%%%%%%%%%%%%%%%%%%%%%%%%%%%%%%%%%%%%%%
\usepackage{amssymb} %Mathematische Bibliotheken
\usepackage{amsmath}
\usepackage{bm}

\usepackage{tikz}
\usepackage[european]{circuitikz}

\usepackage[ngerman]{babel}	%Deutsches Sprachpaket

\begin{document}
	\pagestyle{fancy}
	
	\raggedbottom %Inhalt nicht auf ganze seite verteilen
	
	%Autor: Simon Walker
%Version: 1.0
%Datum: 23.07.2020
%Lizenz: CC BY-NC-SA

\begin{figure}
	\centering
	%Autor: Simon Walker
%Version: 1.0
%Datum: 23.07.2020
%Lizenz: CC BY-NC-SA

\begin{tikzpicture}
	\newcommand{\Block}[4][none]{
		\draw[fill=#1] ($(#2)+(0, 0.5)$) rectangle ($(#3)+(0, -0.5)$);
		\node at ($(#2)!0.5!(#3)$) {#4};
	}
	
	\newcommand{\Inte}[3][none]{
		\draw[fill=#1] ($(#2)+(0, 0.5)$) rectangle ($(#3)+(0, -0.5)$);
		\node at ($(#2)!0.5!(#3)$) {$\dfrac{1}{S}$};
	}
	
	\tikzset{Sum/.style={draw, circle, fill=white,inner sep=1pt, minimum size=0.6}} 
	
	\coordinate (n0) at (0, 0);
	\coordinate (n1) at (0.5, 0);
	\coordinate (n3) at (7, 0);
	
	\coordinate (n10) at (1,0);
	\coordinate (n11) at (2,0);
	\coordinate (n13) at (3,0);
	\coordinate (n14) at (4,0);
	\coordinate (n15) at (4.5,0);
	\coordinate (n16) at (5,0);
	\coordinate (n17) at (6,0);
	
	%Sumatoren
	\node[Sum] (n12) at (2.5,0){$+$};
	\node[Sum] (n2) at (6.5,0){$+$};
	
	%Integrator
	\Inte{n13}{n14}
	
	%Blöcke
	\Block[blue!30!white]{n10}{n11}{$\mathbf{B}$}
	\Block[yellow!30!white]{$(n13)+(0,1.5)$}{$(n14)+(0,1.5)$}{$\mathbf{D}$}
	\Block[red!20!white]{$(n13)+(0,-1.5)$}{$(n14)+(0,-1.5)$}{$\mathbf{A}$}
	\Block[green!30!white]{n16}{n17}{$\mathbf{C}$}
	
	\draw[->] node[left] {$\underline{u}(t)$} (n0) -- (n10);
	\draw[->] (n11) -- (n12);
	\draw[->] (n12) -- (n13);
	\draw[->] (n14) -- (n16);
	\draw[->] (n17) -- (n2);
	\draw[->] (n2) -- (n3) node[right] {$\underline{y}(t)$};
	
	\draw[->] (n1) to[short,*-] ($(n1)+(0, 1.5)$) -- ($(n13)+(0,1.5)$);
	\draw[->] ($(n14)+(0,1.5)$) -| (n2);
	
	\draw[->] (n15) to[short,*-] ($(n15)+(0, -1.5)$) -- ($(n14)+(0,-1.5)$);
	\draw[->] ($(n13)+(0,-1.5)$) -| (n12);

\end{tikzpicture}

\end{figure}

%Einleitung
Bei diesem Beispiel handelt es sich um ein SISO (singel Input, singel Output) System und hat zwei Zustände.

%Matrizen
\begin{equation*}
	\colorbox{red!20!white}{$
	\mathbf{A} = \left[\begin{array}{cc}
		A_{11} & A_{12}\\
		A_{21} & A_{22}
	\end{array}\right]
	$}
	\qquad
	\colorbox{blue!30!white}{$
	\mathbf{B} = \left[\begin{array}{c}
		B_{1}\\
		B_{2}
	\end{array}\right]
	$}
	\qquad
	\colorbox{green!30!white}{$
	\mathbf{C} = \left[\begin{array}{cc}
		C_{1} & C_{2}
	\end{array}\right]
	$}
	\qquad
	\colorbox{yellow!30!white}{$
	\mathbf{D} = D_1
	$}
\end{equation*}

\begin{figure}[h]
	\centering
	%Autor: Simon Walker
%Version: 1.0
%Datum: 23.07.2020
%Lizenz: CC BY-NC-SA

\begin{tikzpicture}
	\newcommand{\Vers}[4][none]{
		\draw[fill=#1] ($(#2)+(0, 0.4)$) -- (#3) -- ($(#2)+(0, -0.4)$) -- cycle;
		\node at ($(#2)!0.3!(#3)$) {#4};
	}
	
	\newcommand{\Inte}[3][none]{
		\draw[fill=#1] ($(#2)+(0, 0.5)$) rectangle ($(#3)+(0, -0.5)$);
		\node at ($(#2)!0.5!(#3)$) {$\dfrac{1}{S}$};
	}
	
	\tikzset{Sum/.style={draw, circle, fill=white,inner sep=1pt, minimum size=0.6}}
	
	%Koordinaten
	\coordinate (n0) at (-0.3, 0);
	\coordinate (n2) at (11.3, 0);
	
	\coordinate (n10) at (1, -3.5);
	\coordinate (n11) at (2, -3.5);
	\coordinate (n12) at (3, -3.5);
	\coordinate (n13) at (4, -3.5);
	\coordinate (n14) at (5, -3.5);
	\coordinate (n15) at (6, -3.5);
	\coordinate (n16) at (7, -3.5);
	\coordinate (n17) at (8, -3.5);
	\coordinate (n18) at (9, -3.5);
	\coordinate (n19) at (10, -3.5);
	
	\coordinate (n20) at (1, 0);
	\coordinate (n21) at (2, 0);
	\coordinate (n22) at (3, 0);
	\coordinate (n23) at (4, 0);
	\coordinate (n24) at (5, 0);
	\coordinate (n25) at (6, 0);
	\coordinate (n26) at (7, 0);
	\coordinate (n27) at (8, 0);
	\coordinate (n28) at (9, 0);
	
	\coordinate (n30) at (1, 2);
	\coordinate (n31) at (5, 2);
	\coordinate (n32) at (6, 2);
	\coordinate (n33) at (10, 2);
	
	\coordinate (n131) at (4, -2.4);
	\coordinate (n132) at (5, -2.4);
	\coordinate (n133) at (6, -2.4);
	\coordinate (n134) at (6.5, -2.4);
	\coordinate (n135) at (4, -4.6);
	\coordinate (n136) at (5, -4.6);
	\coordinate (n137) at (6, -4.6);
	\coordinate (n138) at (7, -4.6);
	
	\coordinate (n231) at (4, -1.1);
	\coordinate (n232) at (5, -1.1);
	\coordinate (n233) at (6, -1.1);
	\coordinate (n234) at (6.5, -1.1);
	\coordinate (n235) at (4, 1.1);
	\coordinate (n236) at (5, 1.1);
	\coordinate (n237) at (6, 1.1);
	\coordinate (n238) at (7, 1.1);
	
	
	%Sumatoren
	\node[Sum] (n23) at (4,0){$+$};
	\node[Sum] (n13) at (4,-3.5){$+$};
	\node[Sum] (n1) at (10, 0){$+$};
	
	%Verstärker
	\Vers[yellow!30!white]{n31}{n32}{$D_1$}
	\Vers[blue!30!white]{n21}{n22}{$B_1$}
	\Vers[blue!30!white]{n11}{n12}{$B_2$}
	\Vers[green!30!white]{n27}{n28}{$C_1$}
	\Vers[green!30!white]{n17}{n18}{$C_2$}
	
	\Vers[red!20!white]{n133}{n132}{$A_{12}$}
	\Vers[red!20!white]{n137}{n136}{$A_{22}$}
	\Vers[red!20!white]{n233}{n232}{$A_{21}$}
	\Vers[red!20!white]{n237}{n236}{$A_{11}$}
	
	%Integratorten
	\Inte{n14}{n15}
	\Inte{n24}{n25}
	
	
	\draw (n0) node[above right] {$u(t)$} to[short, -*] (n20);
	
	\draw[->] (n20) -- (n30) -- (n31);
	\draw[->] (n32) -- (n33) --(n1);
	
	\draw[->] (n20) -- (n21);
	\draw[->] (n22) -- (n23);
	\draw[->] (n23) -- (n24);
	\draw[->] (n25) -- (n27);
	\draw[->] (n28) -- (n1);
	
	\draw[->] (n20) -- (n10) -- (n11);
	\draw[->] (n12) -- (n13);
	\draw[->] (n13) -- (n14);
	\draw[->] (n15) -- (n17);
	\draw[->] (n18) -- (n19) -- (n1);
	
	\draw (n26) to[short, *-] (n238);
	\draw[->] (n238) -- (n237);
	\draw[->] (n236) -- (n235) -- (n23);
	
	\draw (n26) to[short, *-] (n134);
	\draw[->] (n134) -- (n133);
	\draw[->] (n132) -- (n131) -- (n13);
	
	\draw (n16) to[short, *-] (n234);
	\draw[->] (n234) -- (n233);
	\draw[->] (n232) -- (n231) -- (n23);
	
	\draw (n16) to[short, *-] (n138);
	\draw[->] (n138) -- (n137);
	\draw[->] (n136) -- (n135) -- (n13);
	
	\draw[->] (n1) -- (n2) node[above left] {$y(t)$};
\end{tikzpicture}

\end{figure}

	
\end{document}
